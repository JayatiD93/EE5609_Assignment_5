\documentclass[journal,12pt,twocolumn]{IEEEtran}
%
\usepackage{setspace}
\usepackage{gensymb}
%\doublespacing
\singlespacing

%\usepackage{graphicx}
%\usepackage{amssymb}
%\usepackage{relsize}
\usepackage[cmex10]{amsmath}
%\usepackage{amsthm}
%\interdisplaylinepenalty=2500
%\savesymbol{iint}
%\usepackage{txfonts}
%\restoresymbol{TXF}{iint}
%\usepackage{wasysym}
\usepackage{amsthm}
%\usepackage{iithtlc}
\usepackage{mathrsfs}
\usepackage{txfonts}
\usepackage{stfloats}
\usepackage{bm}
\usepackage{cite}
\usepackage{cases}
\usepackage{subfig}
%\usepackage{xtab}
\usepackage{longtable}
\usepackage{multirow}
%\usepackage{algorithm}
%\usepackage{algpseudocode}
\usepackage{enumitem}
\usepackage{mathtools}
\usepackage{steinmetz}
\usepackage{tikz}
\usepackage{circuitikz}
\usepackage{verbatim}
\usepackage{tfrupee}
\usepackage[breaklinks=true]{hyperref}
%\usepackage{stmaryrd}
\usepackage{tkz-euclide} % loads  TikZ and tkz-base
%\usetkzobj{all}
\usetikzlibrary{calc,math}
\usepackage{listings}
    \usepackage{color}                                            %%
    \usepackage{array}                                            %%
    \usepackage{longtable}                                        %%
    \usepackage{calc}                                             %%
    \usepackage{multirow}                                         %%
    \usepackage{hhline}                                           %%
    \usepackage{ifthen}                                           %%
  %optionally (for landscape tables embedded in another document): %%
    \usepackage{lscape}     
\usepackage{multicol}
\usepackage{chngcntr}
%\usepackage{enumerate}

%\usepackage{wasysym}
%\newcounter{MYtempeqncnt}
\DeclareMathOperator*{\Res}{Res}
%\renewcommand{\baselinestretch}{2}
\renewcommand\thesection{\arabic{section}}
\renewcommand\thesubsection{\thesection.\arabic{subsection}}
\renewcommand\thesubsubsection{\thesubsection.\arabic{subsubsection}}

\renewcommand\thesectiondis{\arabic{section}}
\renewcommand\thesubsectiondis{\thesectiondis.\arabic{subsection}}
\renewcommand\thesubsubsectiondis{\thesubsectiondis.\arabic{subsubsection}}

% correct bad hyphenation here
\hyphenation{op-tical net-works semi-conduc-tor}
\def\inputGnumericTable{}                                 %%

\lstset{
%language=C,
frame=single, 
breaklines=true,
columns=fullflexible
}
%\lstset{
%language=tex,
%frame=single, 
%breaklines=true
%}

\begin{document}
%


\newtheorem{theorem}{Theorem}[section]
\newtheorem{problem}{Problem}
\newtheorem{proposition}{Proposition}[section]
\newtheorem{lemma}{Lemma}[section]
\newtheorem{corollary}[theorem]{Corollary}
\newtheorem{example}{Example}[section]
\newtheorem{definition}[problem]{Definition}
%\newtheorem{thm}{Theorem}[section] 
%\newtheorem{defn}[thm]{Definition}
%\newtheorem{algorithm}{Algorithm}[section]
%\newtheorem{cor}{Corollary}
\newcommand{\BEQA}{\begin{eqnarray}}
\newcommand{\EEQA}{\end{eqnarray}}
\newcommand{\define}{\stackrel{\triangle}{=}}

\bibliographystyle{IEEEtran}
%\bibliographystyle{ieeetr}


\providecommand{\mbf}{\mathbf}
\providecommand{\pr}[1]{\ensuremath{\Pr\left(#1\right)}}
\providecommand{\qfunc}[1]{\ensuremath{Q\left(#1\right)}}
\providecommand{\sbrak}[1]{\ensuremath{{}\left[#1\right]}}
\providecommand{\lsbrak}[1]{\ensuremath{{}\left[#1\right.}}
\providecommand{\rsbrak}[1]{\ensuremath{{}\left.#1\right]}}
\providecommand{\brak}[1]{\ensuremath{\left(#1\right)}}
\providecommand{\lbrak}[1]{\ensuremath{\left(#1\right.}}
\providecommand{\rbrak}[1]{\ensuremath{\left.#1\right)}}
\providecommand{\cbrak}[1]{\ensuremath{\left\{#1\right\}}}
\providecommand{\lcbrak}[1]{\ensuremath{\left\{#1\right.}}
\providecommand{\rcbrak}[1]{\ensuremath{\left.#1\right\}}}
\theoremstyle{remark}
\newtheorem{rem}{Remark}
\newcommand{\sgn}{\mathop{\mathrm{sgn}}}
\providecommand{\abs}[1]{\left\vert#1\right\vert}
\providecommand{\res}[1]{\Res\displaylimits_{#1}} 
\providecommand{\norm}[1]{\left\lVert#1\right\rVert}
%\providecommand{\norm}[1]{\lVert#1\rVert}
\providecommand{\mtx}[1]{\mathbf{#1}}
\providecommand{\mean}[1]{E\left[ #1 \right]}
\providecommand{\fourier}{\overset{\mathcal{F}}{ \rightleftharpoons}}
%\providecommand{\hilbert}{\overset{\mathcal{H}}{ \rightleftharpoons}}
\providecommand{\system}{\overset{\mathcal{H}}{ \longleftrightarrow}}
	%\newcommand{\solution}[2]{\textbf{Solution:}{#1}}
\newcommand{\solution}{\noindent \textbf{Solution: }}
\newcommand{\cosec}{\,\text{cosec}\,}
\providecommand{\dec}[2]{\ensuremath{\overset{#1}{\underset{#2}{\gtrless}}}}
\newcommand{\myvec}[1]{\ensuremath{\begin{pmatrix}#1\end{pmatrix}}}
\newcommand{\mydet}[1]{\ensuremath{\begin{vmatrix}#1\end{vmatrix}}}
%\numberwithin{equation}{section}
\numberwithin{equation}{subsection}
%\numberwithin{problem}{section}
%\numberwithin{definition}{section}
\makeatletter
\@addtoreset{figure}{problem}
\makeatother

\let\StandardTheFigure\thefigure
\let\vec\mathbf
%\renewcommand{\thefigure}{\theproblem.\arabic{figure}}
\renewcommand{\thefigure}{\theproblem}
%\setlist[enumerate,1]{before=\renewcommand\theequation{\theenumi.\arabic{equation}}
%\counterwithin{equation}{enumi}


%\renewcommand{\theequation}{\arabic{subsection}.\arabic{equation}}

\def\putbox#1#2#3{\makebox[0in][l]{\makebox[#1][l]{}\raisebox{\baselineskip}[0in][0in]{\raisebox{#2}[0in][0in]{#3}}}}
     \def\rightbox#1{\makebox[0in][r]{#1}}
     \def\centbox#1{\makebox[0in]{#1}}
     \def\topbox#1{\raisebox{-\baselineskip}[0in][0in]{#1}}
     \def\midbox#1{\raisebox{-0.5\baselineskip}[0in][0in]{#1}}

\vspace{3cm}


\title{Assignment 5}
\author{Jayati Dutta}





% make the title area
\maketitle

\newpage

%\tableofcontents

\bigskip

\renewcommand{\thefigure}{\theenumi}
\renewcommand{\thetable}{\theenumi}
%\renewcommand{\theequation}{\theenumi}


\begin{abstract}
This is a simple document explaining how to prove the congruence of two triangles.
\end{abstract}

%Download all python codes 
%
%\begin{lstlisting}
%svn co https://github.com/JayatiD93/trunk/My_solution_design/codes
%\end{lstlisting}

Download all and latex-tikz codes from 
%
\begin{lstlisting}
svn co https://github.com/gadepall/school/trunk/ncert/geometry/figs
\end{lstlisting}
%


\section{Problem}
$\triangle ABC$ and $\triangle DBC$ are two isosceles triangles on the same base $BC$ and the vertices $A$ and $D$ are on the same side of $BC$. If $AD$ is extended to intersect $BC$ at $P$, show that \\
a) $\triangle ABD$ $\cong$ $\triangle ACD$ \\
b) $\triangle ABP$ $\cong$ $\triangle ACP$ \\
c) $AP$ bisects $\angle A$ as well as $\angle D$\\
d) $AP$ is the parpendicular bisector of $BC$

\section{Explanation}
\begin{figure}[!ht]
\centering
\resizebox{\columnwidth}{!}{\input{./figs/triangle.tex}}
\caption{Iso-sceles Triangles by Latex-Tikz}
\label{fig:iso_scelen}	
\end{figure}

The above problem statement is depicted in the figure \ref{fig:iso_scelen} where the vertices are: A, B and C for $\triangle ABC$ and D, B and C for $\triangle DBC$. For $\triangle ABC$ the sides AB, BC and CA are represented by the vectors $\vec{A-B}$ , $\vec{B-C}$ and $\vec{C-A}$ and for $\triangle DBC$ the sides DB, BC and CD are represented by $\vec{D-B}$ , $\vec{B-C}$ and $\vec{C-D}$.

From the problem statement we get that:
\begin{equation}
\begin{aligned}
\norm{\vec{A}-\vec{B}} = \norm{\vec{A}-\vec{C}}
%\implies \norm{\vec{AB}} = \norm{\vec{AC}}
\end{aligned}
\end{equation}
\label{cond1}
\begin{align}
\norm{\vec{D}-\vec{B}} = \norm{\vec{D}-\vec{C}}
%\implies \norm{\vec{DB}} = \norm{\vec{DC}}
\end{align}
\label{cond2}
\begin{align}
\vec{P-C} = K_2 (\vec{B-C})\\
\vec{P-B} = K_1 (\vec{B-C})
\end{align}
As the $\triangle ABC$ is an iso-scelen triangle,
\begin{align}
\angle ABC = \angle ACB =\gamma
\end{align}
Similarly, as the $\triangle DBC$ is an iso-scelen triangle,
\begin{align}
\angle DBC = \angle DCB 
\end{align}
Let consider $\angle APB$ = $\phi_1$, $\angle APC$ = $\phi_2$ and $\phi_1 + \phi_2$ = 180 $\degree$.
From the triangular law of vector addition, we can also get that:
\begin{align}
\vec{A-B} = (\vec{A-P}) + (\vec{P-B})\\
\vec{A-C} = (\vec{A-P}) + (\vec{P-C})\\
\vec{D-B} = (\vec{D-P}) + (\vec{P-B})\\
\vec{D-C} = (\vec{D-P}) + (\vec{P-C})
\end{align}
Now squaring both side of equation \ref{cond1}, 
\begin{align}
\norm{\vec{A}-\vec{B}} =\norm{\vec{A}-\vec{C}}\\
\text{or,}\norm{\vec{A}-\vec{B}}^2 = \norm{\vec{A}-\vec{C}}^2\\
\text{or,}\norm{(\vec{A-P}) + (\vec{P-B})}^2 = \norm{(\vec{A-P}) + (\vec{P-C})}^2\\
\text{or,}((\vec{A-P}) + (\vec{P-B}))^T ((\vec{A-P}) + (\vec{P-B})) =\\
 ((\vec{A-P}) + (\vec{P-C}))^T ((\vec{A-P}) + (\vec{P-C}))\\
\text{or,}((\vec{A-P})^T + K_1(\vec{B-C})^T)((\vec{A-P}) + K_1(\vec{B-C})) =\\
((\vec{A-P})^T + K_2(\vec{B-C})^T)((\vec{A-P}) + K_2(\vec{B-C}))\\
\text{or,} (K_1 - K_2)((\vec{B-C})^T (\vec{A-P}) \\
+ (\vec{A-P})^T (\vec{B-C}) + (K_1 + K_2)\norm{\vec{B-C}}^2)=0
\end{align}
So, there are 2 cases, either $(K_1 - K_2)$ = 0 or 
\begin{align}
((\vec{B-C})^T (\vec{A-P}) \\
+ (\vec{A-P})^T (\vec{B-C}) + (K_1 + K_2)\norm{\vec{B-C}}^2)= 0\\
\text{or,}(K_1 + K_2)\norm{\vec{B-C}}^2 = - (\vec{B-C})^T (\vec{A-P})\\
-(\vec{A-P})^T (\vec{B-C})\\
\text{or,}(K_1 + K_2)\frac{\norm{\vec{B-C}}}{\norm{\vec{A-P}}}= \frac{- (\vec{P-B})^T (\vec{A-P})}{\norm{\vec{P-B}} \norm{\vec{A-P}}}\\
 - \frac{(\vec{P-C})^T (\vec{A-P})}{\norm{\vec{P-C}} \norm{\vec{A-P}}}\\
\text{or,}(K_1 + K_2)\frac{\norm{\vec{B-C}}}{\norm{\vec{A-P}}} = - \cos\phi_1 - \cos\phi_2\\
\text{or,}(K_1 + K_2)\frac{\norm{\vec{B-C}}}{\norm{\vec{A-P}}} = - \cos\phi_1 + \cos\phi_1\\
\text{or,}(K_1 + K_2)\frac{\norm{\vec{B-C}}}{\norm{\vec{A-P}}} = 0\\
\implies K_1 = -K_2
\end{align}
So, for the first case, that is, when $K_1$ = $K_2$ :
\begin{align}
\cos\theta_1 = \frac{(\vec{A-B})^T (\vec{A-P})}{\norm{\vec{A-B}}\norm{\vec{A-P}}}\\
\text{or,}\cos\theta_1  = \frac{((\vec{A-P})+(\vec{P-B}))^T (\vec{A-P})}{\norm{\vec{A-B}}\norm{\vec{A-P}}}
\end{align}
Similarly,
\begin{align}
\cos\theta_2 = \frac{(\vec{A-C})^T (\vec{A-P})}{\norm{\vec{A-C}}\norm{\vec{A-P}}}\\
\text{or,}\cos\theta_2  = \frac{((\vec{A-P})+(\vec{P-C}))^T (\vec{A-P})}{\norm{\vec{A-B}}\norm{\vec{A-P}}}\\
\text{or,}\cos\theta_2  = \frac{((\vec{A-P})+(\vec{P-B}))^T (\vec{A-P})}{\norm{\vec{A-B}}\norm{\vec{A-P}}}
\end{align}
So, we can say that, $\theta_1$ = $\theta_2$
Now for $\triangle DBC$,
\begin{align}
\cos\alpha = \frac{(\vec{D-B})^T (\vec{D-P})}{\norm{\vec{D-B}}\norm{\vec{D-P}}}\\
\text{or,}\cos\alpha  = \frac{((\vec{D-P})+(\vec{P-B}))^T (\vec{D-P})}{\norm{\vec{D-B}}\norm{\vec{D-P}}}
\end{align}
Similarly,
\begin{align}
\cos\beta = \frac{(\vec{D-C})^T (\vec{D-P})}{\norm{\vec{D-C}}\norm{\vec{D-P}}}\\
\text{or,}\cos\beta = \frac{((\vec{D-P})+(\vec{P-C}))^T (\vec{D-P})}{\norm{\vec{D-B}}\norm{\vec{D-P}}}\\
\text{or,}\cos\beta  = \frac{((\vec{D-P})+(\vec{P-B}))^T (\vec{D-P})}{\norm{\vec{D-B}}\norm{\vec{D-P}}}
\end{align}
So, we can conclude that $\alpha$ = $\beta$.
These imply that $\vec{A-P}$ bisects $\angle A$ as well as $\angle D$.
For $K_1 = K_2$, $\vec{P-C}$ = $\vec{P-B}$
\begin{align}
\cos\phi_1 = \frac{(\vec{A-P})^T (\vec{P-B})}{\norm{\vec{P-B}}\norm{\vec{A-P}}}\\
\text{or,}\cos\phi_1 = \frac{((\vec{A-B}) + (\vec{B-P}))^T (\vec{P-B})}{\norm{\vec{P-B}}\norm{\vec{A-P}}}\\
\text{or,}\cos\phi_1 = \frac{(\vec{A-B})^T (\vec{P-B})+ (\vec{B-P})^T (\vec{P-B})}{\norm{\vec{P-B}}\norm{\vec{A-P}}}\\
\text{or,}\cos\phi_1 = \cos\gamma - \frac{\norm{\vec{B-P}}}{\norm{\vec{A-P}}}\\
\end{align}
Similarly,
\begin{align}
\cos\phi_2 = \cos\gamma - \frac{\norm{\vec{P-C}}}{\norm{\vec{A-P}}}\\
\text{or,}\cos\phi_2 = \cos\gamma - \frac{\norm{\vec{P-B}}}{\norm{\vec{A-P}}}\\
\text{or,}\cos\phi_2 = \cos\gamma - \frac{\norm{\vec{B-P}}}{\norm{\vec{A-P}}}\\
\end{align} 
From here it can be concluded
\begin{align}
\cos\phi_1=\cos\phi_2\\
\text{or,}\phi_1=\phi_2\\
\text{or,}\phi_1 + \phi_2 = 180 \degree\\
\implies \phi_1 = 90 \degree
\end{align}
For the second case, that is, when $K_1=-K_2$,
$\vec{P-C} = \vec{B-P}$
\begin{align}
\cos\phi_1 = \cos\gamma - \frac{\norm{\vec{B-P}}}{\norm{\vec{A-P}}}\\
\cos\phi_2 = \cos\gamma - \frac{\norm{\vec{P-C}}}{\norm{\vec{A-P}}}\\
\implies \cos\phi_2 = \cos\gamma - \frac{\norm{\vec{B-P}}}{\norm{\vec{A-P}}}\\
\end{align} 
Which also implies that 
\begin{align}
\cos\phi_1=\cos\phi_2\\
\text{or,}\phi_1=\phi_2\\
\text{or,}\phi_1 + \phi_2 = 180 \degree\\
\implies \phi_1 = 90 \degree
\end{align}
Hence, it is proved that AP bisects BC perpendicularly.
\renewcommand{\theequation}{\theenumi}
%\begin{enumerate}[label=\thesection.\arabic*.,ref=\thesection.\theenumi]
%\numberwithin{equation}{enumi}
%\item Verification of the above problem using python code.\\
%%\solution The  following Python code generates Fig. \ref{fig:point_distance}
%%\begin{lstlisting}
%%codes/det_check.py
%%\end{lstlisting}
%
%\end{enumerate}

\end{document}



